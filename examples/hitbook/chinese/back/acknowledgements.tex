% !Mode:: "TeX:UTF-8"
\begin{acknowledgements}
两年半的光景过去,弹指一挥间,我的研究生生涯即将进入尾声。在这段美好的时光里,我感受到了哈工大深圳师生对我的真挚之情。不论是在研一学期的各种专业课程,还是在研二之后的科研生活中,我的逻辑思维水平、分析和解决问题的能力都得到了较大水平的提升,我的人格和品质也在老师的耳濡目染下更加完善。今后,我会时时刻刻谨记母校对我的教诲,将科研中的严谨态度带到之后的工作和生活中,满怀热忱和希望去追求下一阶段的人生目标。

历时一年半的毕业论文已画上句号,我们也即将背负行囊,踏上新的战场继续我们的璀璨人生。海内存知己,天涯若比邻,在这即将挥手告别之际,我首先要感谢我的导师黄荷姣教授。从论文的开题、论文框架的分析到最终的定稿,我的导师黄荷姣教授都花费了宝贵的时间。同时黄老师严谨的治学态度、随和的性格以及处处替学生考虑的品格,都将如黑夜中的灯塔一般深深地影响我之后的工作和生活态度。我也要感谢顾崇林老师,在小论文的撰写过程中,他倾注了大量的心血,不遗余力地帮我梳理论文的行文逻辑。在此谨向黄老师和顾老师致以诚挚的感谢和崇高的敬意。

其次,我要感谢我们科研小组的张硕师兄,从加入TSDM小组至今,张硕师兄助我甚多,他不仅时刻关注我的科研进度,为我的课题提供了大量宝贵的意见和建议,同时在生活上也给予我莫大的帮助。在此,也对张硕师兄表示真诚的谢意。

再次,我还要感谢云计算实验室的其他同学和伙伴,特别是陈孝飞、江桥和张魁元。在和他们的相处过程中,我学会了更好地为人处世和分享喜怒哀乐,让我如此留恋与他们在一起时的谈笑风生,期待今后还会再见。

最后,我要衷心感谢支持我学业的家人和朋友,是他们给了我在学习道路上不断探索的动力和自信,使我能够在知识的跑道上驰骋,也使我的人生得到了升华。

研究生生活虽然即将结束,但是新的征途即将启程,我将带着这份荣光一展宏图,方能不负平生之学。

\end{acknowledgements}
