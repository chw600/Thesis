% !Mode:: "TeX:UTF-8" 
\begin{conclusions}
随着城市现代化发展和传感器设备的普及,城市居民和交通工具的活动轨迹都被传感器检测并记录下来。收集高质量的交通数据并建模其中的隐含信息对建设智能交通应用具有重大的现实意义,比如居民出行计划制定、交通拥塞控制等。然而,由于GPS信号干扰或者人为损坏等原因,传感器采集到的交通数据往往都包含缺失值,这限制了交通应用的决策准确性。因此,合理地处理缺失值是交通数据挖掘的首要任务。现有的基于统计机器学习和深度学习的填充算法能取得较好的补全效果,但也存在明显不足。主要表现在只专注于交通数据时间依赖性或者空间相关性的建模,却缺乏对时空特性的整体考虑和深层次挖掘。另外,之前的方法并没有对不同时间跨度的交通数据进行分类分析。针对以上问题,本文根据不同的时间跨度的场景,利用变分自编码器分别设计了相应的交通缺失值填充算法,且所设计方法实现了较低的填充绝对误差和相对误差。本文的研究工作围绕长短期交通缺失场景展开,具体内容如下:

(1)面向长期交通数据缺失值填充场景。考虑到城市交通网络可以按照经纬度划分为等距离的栅格图,那么在每个固定的时间区间内,该栅格图可视为包含多通道的图片,本文将一个城市内每日按照固定频率采集的栅格数据视为一段由多个连续帧组成的多通道视频,该处理方式能极大程度上保留原始交通数据的时空信息。基于此,本文提出了一种针对长期城市交通数据的生成式填充神经网络模型。该模型基于带有赛尔维斯特标准化流的变分自编码器,通过学习随机隐变量的非高斯分布来高度抽象交通数据的隐含信息。同时,受注意力机制的启发,本文设计了一种能有效抽取鲁棒的时空特征表示的网络模块,它能区分不完整数据中的缺失值和观测值的有效性,并对提取的时空特征图在多个维度上进行校正。在开源交通数据集TaxiBJ、BikeNYC和TaxiNYC的实验结果表明,本文提出的填充模型\textit{ST-VAE}在多种误差指标上都取得了优于主流填充算法的成绩。

(2)面向短期交通数据缺失值填充场景。尽管基于带有标准化流的变分自编码器填充模型ST-VAE能取得很好的填充效果,但是考虑到模型复杂度和推断效率等因素,\textit{ST-VAE}无法针对短期交通填充场景(即填充某一个时间槽的栅格图)进行精细化填充。受交通数据时间周期性的启发,本文设计了一种时空特征解耦的填充模型(\textit{ST-LBAVAE})。首先,该方法利用了轻量级的双向注意力图和跳级连接的变分自编码器结构来达到在编解码阶段动态更新掩膜和多层次空间特征融合的目的。其次,\textit{ST-LBAVAE}充分利用了历史邻近、周期、趋势的数据进行时间相似性的预测。同时,此方法融合了诸如天气、节假日等外部因素来对极端情况下的缺失值填充所需的信息进行补充。最后,本文优化了模型的目标函数,使其不仅能填充输入数据的缺失值,且能保证高层语义信息的准确性。在上文提到的数据集上的实验结果显示,本文设计的模型\textit{ST-LBAVAE}的填充效果优于交通领域主流的算法,能够逐帧地对交通栅格数据进行精细化填充。

尽管本文针对两种不同时间跨度下的城市交通数据缺失场景分别设计了两种算法,且在多个开源交通数据集上取得了不错的填充效果,但是考虑到不同城市的交通情况各不相同,本文仅仅在北京市和纽约市这两个具有代表性的道路系统规划清晰的城市上进行了模型的验证,而考虑如何设计自适应的填充算法或者利用迁移学习等深度学习方法使得模型能在多城市上均能取得良好的效果仍然是一个值得探索的方向。

\end{conclusions}
