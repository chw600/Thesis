% !Mode:: "TeX:UTF-8"

\hitsetup{
  %******************************
  % 注意:
  %   1. 配置里面不要出现空行
  %   2. 不需要的配置信息可以删除
  %******************************
  %
  %=====
  % 秘级
  %=====  ;U495
  statesecrets={公开},
  natclassifiedindex={TP183},
  intclassifiedindex={004.8},
  % 
  %=========
  % 中文信息
  %=========
%  ctitleone={局部多孔质气体静压},%本科生封面使用
%  ctitletwo={轴承关键技术的研究},%本科生封面使用
  ctitlecover={基于变分自编码器的交通数据填充算法设计},%放在封面中使用,自由断行
  ctitle={基于变分自编码器的交通数据填充算法设计},%放在原创性声明中使用
%  csubtitle={一条副标题}, %一般情况没有,可以注释掉
  cxueke={工程},
  csubject={计算机技术},
  caffil={哈尔滨工业大学(深圳)},
  cauthor={陈锦毅},
  csupervisor={黄荷姣教授},
%  cassosupervisor={某某某教授}, % 副指导老师
%  ccosupervisor={某某某教授}, % 联合指导老师
  % 日期自动使用当前时间,若需指定按如下方式修改:
  cdate={2022年12月},
  cstudentid={20S151116},
  cstudenttype={专业学位论文}, %非全日制教育申请学位者
%  cnumber={no9527}, %编号
%  cpositionname={哈铁西站}, %博士后站名称
%  cfinishdate={20XX年X月---20XX年X月}, %到站日期
%  csubmitdate={20XX年X月}, %出站日期
%  cstartdate={3050年9月10日}, %到站日期
%  cenddate={3090年10月10日}, %出站日期
  %(同等学力人员)、(工程硕士)、(工商管理硕士)、
  %(高级管理人员工商管理硕士)、(公共管理硕士)、(中职教师)、(高校教师)等
  %
  %
  %=========
  % 英文信息
  %=========
  etitle={Design of Traffic Data Filling Algorithm Based on Variational Autoencoder},
%  esubtitle={This is the sub title},
  exueke={Engineering},
  esubject={Computer Applied Technology},
  eaffil={\emultiline[t]{Harbin Institute of Technology, Shenzhen}},
  eauthor={Jinyi Chen},
  esupervisor={Prof. Hejiao Huang},
%  eassosupervisor={XXX},
  % 日期自动生成,若需指定按如下方式修改:
  edate={December, 2022},
  estudenttype={Master of Engineering},
  %
  % 关键词用“英文逗号”分割
  ckeywords={智能交通, 时空数据, 数据填充, 变分自编码器, 注意力机制},
  ekeywords={intelligent transportation, spatio-temporal data, data imputation, variational autoencoder, attention mechanism},
}

\begin{cabstract}

%摘要的字数(以汉字计),硕士学位论文一般为500 $\sim$ 1000字,博士学位论文为1000 $\sim$ 2000字,
%均以能将规定内容阐述清楚为原则,文字要精练,段落衔接要流畅。摘要页不需写出论文题目。
%英文摘要与中文摘要的内容应完全一致,在语法、用词上应准确无误,语言简练通顺。
%留学生的英文版博士学位论文中应有不少于3000字的“详细中文摘要”。
%
%  关键词是为了文献标引工作、用以表示全文主要内容信息的单词或术语。关键词不超过 5
%  个,每个关键词中间用分号分隔。(模板作者注:关键词分隔符不用考虑,模板会自动处
%  理。英文关键词同理。)
随着人们日常出行的需求提高,现阶段被传感器所检测到的交通数据逐渐增多。智能交通系统可以从这些交通数据中提取更为丰富的信息,这将提升它在相关场景的应用性能,比如更有效地帮助居民制定出行计划等。但是,检测器故障或信号干扰,会给交通数据的维护带来问题,比如,交通数据中存在缺失值等,这些问题将限制了交通应用的决策准确性。因此,合理地处理缺失值是交通数据挖掘的首要任务。现有的基于统计机器学习和深度学习的填充算法能取得较好的补全效果,但也存在明显不足。首先,周期依赖性是交通数据的一个重要特征,即交通数据在不同的时间间隔下反映了各种时间特征,但是大多数模型没有考虑这一特征。其次,大多数模型忽略了交通数据计算中缺失值的影响。针对上述问题,主要完成了以下几个方面的研究。

考虑时间缺失值分布特性的填充算法设计,我们将交通数据按照不同时间间隔(例如,每小时、每天和每周)划分视频数据研究交通数据填充问题。同时,我们提出一个新模型 MTSVAE。MTSVAE 可用于学习潜在变量的分布,这个特性帮助其在处理不同缺失率的交通数据时表现强鲁棒性。特征编码模块是 MTSVAE 的一个子模块,它增强了 MTSVAE 提取不完整数据的时空特征的能力。在特征编码模块中,BiConvGRUI模块被用来刻画不完整交通数据的复杂时间分布。多层 2D 门控卷积被用来捕获交通数据的空间全局特征。最后,双自注意机制进一步提取交通数据的动态时空相关性。在 TaxiBJ 等三个开源交通数据集上的实验结果表明,考虑时间缺失值分布特性的填充算法在填充效果上优于其他交通数据填充算法。

考虑空间缺失值分布特性的填充算法设计。在第一部分工作的基础上,进一步考虑不同时间窗的交通数据间的时间相关性,获取更为丰富的时间相关行特征。同时,利用轻量级的可学习双向注意力图来达到掩膜更新和不同尺度的空间特征融合的目的。除此以外,考虑在普通的变分自编码器的基础上,使用拉普拉斯分布替换普通的高斯分布,提升模型抗干扰的能力。同时,改进了模型的目标函数。真实数据集下的实验显示,该算法能够逐帧地对短时间段内的交通栅格数据进行精细化填充。
\end{cabstract}

\begin{eabstract}
With the increasing demand for daily travel, the traffic data detected by sensors is gradually increasing at this stage. Intelligent transportation systems can extract richer information from these traffic data, which will improve its application performance in related scenarios, such as helping residents make travel plans more effectively. However, detector failure or signal interference will bring problems to the maintenance of traffic data, such as missing values ​​in traffic data, etc. These problems will limit the accuracy of decision-making in traffic applications. Therefore, handling missing values ​​reasonably is the primary task of traffic data mining. The existing filling algorithms based on statistical machine learning and deep learning can achieve good completion effects, but they also have obvious shortcomings. First, cycle dependence is an important feature of traffic data, that is, traffic data reflects various temporal features at different time intervals, but most models do not consider this feature. Second, most models ignore the effect of missing values ​​in the computation of traffic data. Aiming at the above problems, the following researches have been mainly completed.

Considering the design of the filling algorithm considering the distribution characteristics of temporal missing values, we divide the traffic data into different time intervals (eg, hourly, daily, and weekly) to study the traffic data filling problem. At the same time, we propose a new model MTSVAE. MTSVAE can be used to learn the distribution of latent variables, and this feature helps it to be robust when dealing with traffic data with different missing rates. The feature encoding module is a sub-module of MTSVAE, which enhances the ability of MTSVAE to extract spatiotemporal features of incomplete data. In the feature encoding module, the BiConvGRUI module is used to characterize the complex temporal distribution of incomplete traffic data. Multilayer 2D gated convolutions are used to capture the spatial global features of traffic data. Finally, a dual self-attention mechanism further extracts the dynamic spatiotemporal correlations of the traffic data. The experimental results on three open source traffic datasets such as TaxiBJ show that the filling algorithm considering the distribution characteristics of temporal missing values ​​is better than other traffic data filling algorithms.

For the traffic scenario of short-term missing data imputation, this thesis proposes an imputation method for decoupling spatial-temporal pattern. First, this method integrates lightweight learnable bidirectional attention maps into a skip-connected variational autoencoder to achieve the purpose of mask-updating and fusion of different scales of spatial features. Second, this method fully considers the impact of multi-scale traffic historical data on the improvement of imputation accuracy. Meanwhile, this method introduces external factors to supplement the information needed for the filling of missing values in extreme cases. Finally, the objective function of the model is improved. Experiments demonstrate that this method can finely impute the traffic raster data in a short period of time frame by frame.
\end{eabstract} 
